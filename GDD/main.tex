\documentclass[a4paper]{report}
\usepackage[utf8]{inputenc}
\usepackage{graphicx}
\usepackage{pdfpages}
\usepackage{lmodern}
\usepackage{txfonts} %Times New Roman
\usepackage[T1]{fontenc}
 
\title{Rusty Steam}
\author{Martin Strapko \\ Peter Šulík}
\date{\today}
\begin{document}
\maketitle

\tableofcontents
 
\chapter{Story}
Zatiaľ nie je k dispozicií
 
\chapter{Feature Listing}
\begin{itemize}
  \item 3D First Person Shooter
  \item Hooks
  \item Enviroment Control
  \item Multiplayer
  \item Weapon Customization
\end{itemize}

\chapter{Gameplay}
\section{Game Mechanics}
 Hlavnou mechanikou hry je priťahovanie a odťahovanie. Hráči pomocou zbraní môžu odhadzovať pohyblivé časti prostredia vrátane iných hráčov. Pri použití voči statickým prvkom prostredia je automaticky použitá opačná sila na hráča - teda napríklad hráč, ktorý chcel odhodiť nepohyblivú stenu sa od nej odhodí. Hráči sa môžu klasicky pohybovať po prostredí a majú určitú formu ukazatela životov. Tie možno nepriatelom ubrať pomocou viacerých možných spôsobov buď polohovaním nepriateľa do určitých zón, trafením pohyblivým predmetom alebo pomocou headshotu - vid nižšie.  

\section{Game Mode}
Hodnotenie hráčov záleži od zvoleného herného módu, aj keď zabíjanie hráčov je dôležité v oboch módoch, jeho dôležitosť sa mení.
\subsection{Team Deathmatch}
Hráči musia nazbierať stanovený počet fragov skôr ako nepriateľský tím. Tie získajú zabitím nepriateľov pomocou prostredia, vyhodenia z mapy alebo spoluprácou so svojími spoluhráčmi na nejakom combe.

\subsection{Capture The Flag}
Skóre tímov sa počíta v množstve vlajok, ktoré odniesli z nepriateľskej základne do svojej. Vlajka z človeka spadne ak bude trafení nejakým objektom (toto trafenie nemusí byť smrteľné) alebo zabití.
 
\section{Controls}
Hráči sa budú vo svete pohybovať pomocou pohybových kláves, štandardne WSAD a strielať so zbraní pomocou tláčítok myši. Taktiež budú mať k dispozícií nejaké utility itemy, ktoré budú môcť štandardne aktivovať pomocou Q a E.

\section{Combos}
Mechaniky hry umožnia hráčom v multiplayeri robiť kombá, ako vystreliť hráča ako projektil alebo takzvaný headshot, ak 2 rôzný ľudia pritiahnú toho istého človeka.

\section{Drops to Powerups}
Vo svete sa budú spawnovať powerupy, ktoré dočasne vylepšia atribúty hráča alebo jeho zbraní. Rýchlejšie projektily, pohyb a tak podobne.

\section{Weapon Customization}
Z menu budú mať hráči možnosť poupravovať si zbrane podľa vlastného štýlu hrania a chute. Budú mať maximálny počet bodov, ktoré budú mocť rozdeliť medzi rôzne parametre zbrane ako dostrel, rýchlosť projektilu, šírka projektilu a ďalšie.
 
\chapter{Visual}
\section{Art Style}
Hlavnou grafickou témou hry bude steampunk, a všetky prvky hry budú ladené podľa tejto témy.
 
\section{User Interface}
\subsection{Menus}
Hráči budú mať možnosť vytvoriť server, pripojiť sa na server, nastaviť systemové parametre (rozlišenie, hlasitosť, senzitivita) ako aj customizovať si niektoré parametre zbraní podľa vlastnej chute.
 
\subsection{Gameplay}
Počas hry budú vidiet indikátory zobrazujúce pripravenosť zbraní, skóre oboch tímov a 3D radar na ktorom sa im bude zobrazovať poloha spojencov a nepriateľov.
 
\end{document}
